\section{Critérios de Avaliação}

A avaliação do curso será feita através de 2 (duas) \textbf{provas práticas},
\textbf{testes teóricos} e \textbf{listas de exercícios}, a
serem realizadas/entregues em datas determinadas no cronograma na plataforma
Moodle ou na plataforma \textit{CodeSchool}.

\subsection{Provas}

As provas serão realizadas com auxílio do computador no laboratório de informática. Cada prova $P_i$ será pontuada em um total de 10 pontos.

No final do semestre será aplicada uma \textbf{prova substitutiva}, cujo resultado \textbf{substituirá} o \textbf{pior} resultado dentre os obtidos nas duas provas práticas, \textbf{independentemente} do resultado da prova substitutiva. Todos os alunos podem fazer a prova substitutiva, se assim desejarem.

\subsection{Testes}

A fim de fortalecer os conceitos teóricos e fundamentais da disciplina, serão aplicados $N$ \textbf{testes teóricos} em sala de aula. Estes testes não possuem data pré-definida para ocorrer e normalmente consistem em um único exercício aplicado ao fim da aula. Cada teste $T_i$ será pontuada em um total de 10 pontos.

\subsection{Listas}

As listas de problemas serão compostas por uma série de problemas relacionados aos tópicos da ementa do curso. O aluno deverá submeter as soluções destes problemas via plataforma Moodle ou via \textit{CodeSchool}, em procedimento a ser detalhado ao longo do curso.

A cada lista $L_i$ será atribuída uma nota na escala de 0 (zero) a 10 (dez) pontos.

\subsection{Menção Final}

A nota final do curso $N_F$ é composta pela nota das provas práticas ($N_P$), pela nota dos testes teóricos ($N_T$) e pela nota das listas ($N_L$).

A nota das provas práticas $N_P$ é dada por:
\[
    N_P = \frac{P_1 + P_2}{2}
\]

A nota dos testes teóricos $N_T$ é dada por:
\[
    N_T = \frac{1}{N} \sum_{i = 1}^N T_i
\]

Onde $N$ é o número de testes aplicados, para o cálculo da nota final dos testes ($N_T$), será excluída a menor nota ou uma falta. A nota das listas $N_L$ é dada por:
\[
	N_L = \frac{1}{N} \sum_{i = 1}^N L_i
\]

Onde $N$ é o número de listas aplicadas. A nota final do curso será então dada pela expressão:
\[
	N_F = 0,7\cdot N_P + 0,15\cdot N_T + 0,15\cdot N_L
\]

A menção final do curso é dada pela nota final $N_F$, de acordo com a tabela abaixo:

\begin{center}
\begin{tabularx}{0.6\textwidth}{ccX}
\toprule
$\mathbf{N_F}$ & \textbf{Menção} & \textbf{Descrição} \\
\toprule
\rowcolor[gray]{.9}
0,0 & \texttt{SR} & \textit{Sem rendimento} \\
de 0,1 a 2,9 & \texttt{II} & \textit{Inferior} \\
\rowcolor[gray]{.9}
de 3,0 a 4,9 & \texttt{MI} & \textit{Médio Inferior} \\
de 5,0 a 6,9 & \texttt{MM} & \textit{Médio} \\
\rowcolor[gray]{.9}
de 7,0 a 8,9 & \texttt{MS} & \textit{Médio Superior} \\
9,0 ou maior & \texttt{SS} & \textit{Superior} \\
\bottomrule
\end{tabularx}
\end{center}

\subsection{Critérios de aprovação}

Obterá \textbf{aprovação} no curso o aluno que cumprir as \textbf{duas} exigências abaixo:

\begin{enumerate}
	\item Ter presença em 75\% ou mais das aulas;
	\item Obter menção final igual ou superior a \texttt{MM}.
\end{enumerate}
